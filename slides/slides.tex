\documentclass{beamer}
\usetheme{Dresden}
\usecolortheme{orchid}

\usepackage[french]{babel}
\usepackage[utf8]{inputenc}
\usepackage[T1]{fontenc}
\usepackage{graphicx}
\usepackage{tikz}
\usepackage{fontawesome}
\usepackage{amsmath}
\usepackage{algorithm}
\usepackage{algpseudocode}

\graphicspath{{../report/ims/}{ims/}}

\newcommand{\N}{\mathbb{N}}

\title{Vérification de la perspective dans les peintures}
\subtitle{Stage de M1}
\date{4 Septembre 2019}
\author{
	Yoann Coudert--Osmont \inst{1} \newline
	\textit{Encadré par} Elmar Eisemann \textit{et} Ricardo Marroqium \inst{2}
}
\institute{
	\inst{1} ENS de Lyon \and
	\inst{2} TU Delft
}

\begin{document}
	
	\begin{frame}
		\titlepage
	\end{frame}

	\section{Règles de perspective}

	\begin{frame}{Lignes et points de fuites}
		\centering
		\begin{figure}
			\begin{tikzpicture}[thick, scale=1.2]
			\draw[] (0, 0) -- (1, 0) -- (1, 1) -- (0, 1) -- cycle;
			\draw[red] (-2.5, 2.25) -- (6, 2.25);
			\draw[green] (0, 1) -- (0.8, 1.5);
			\draw[green, dashed] (0.8, 1.5) -- (2, 2.25);
			\draw[green] (1, 1) -- (1.4, 1.5);
			\draw[green, dashed] (1.4, 1.5) -- (2, 2.25);
			\draw (0.8, 1.5) -- (1.4, 1.5);
			\draw[green] (1, 0) -- (1.4, 0.9);
			\draw[green, dashed] (1.4, 0.9) -- (2, 2.25);
			\draw (1.4, 0.9) -- (1.4, 1.5);
			
			\node[blue] (VP) at (2, 2.25) {$\bullet$};
			\node[above, blue] at (2, 2.25) {\small Point de fuite};
			\node[below, red] at (5, 2.25) {\small Ligne de fuite};
			\node[left, green] at (0.4, 1.4) {\small Lignes parallèles};
			\end{tikzpicture}
		\end{figure}
	\end{frame}

	\begin{frame}
		\centering
		\begin{figure}[h]
			\begin{tikzpicture}[thick, yscale=0.65, xscale=0.82]
			\draw[very thick] (0, 6) -- (12, 6);
			\draw[semithick] (0, 0) -- (12, 0);
			\node[above] at (6, 6) {\small Ligne de fuite};
			\node[above, red] at (0, 6) {\small $A$};
			\node[above, blue] at (4, 6) {\small $B$};
			\node[above, green] at (12, 6) {\small $C$};
			
			\foreach \i in {2, ..., 11} \draw[red] (0, 6) -- (\i, 0);
			\foreach \i in {1.333, 2.666, ..., 9.333} \draw[blue] (4, 6) -- (\i, 0);
			
			\draw[red] (2, -0.3) node {\texttt{|}} -- node[below] {$d_A$} (3, -0.3) node {\texttt{|}};
			\draw[blue] (4, -0.3) node {\texttt{|}} -- node[below] {$d_B$} (5.333, -0.3) node {\texttt{|}};
			\draw[green] (6, -0.3) node {\texttt{|}} -- node[below] {$d_C$} (8, -0.3) node {\texttt{|}};
			
			\node[below, align=center] at (10.2, 0)
			{\footnotesize ligne parallèle à \\[-1mm]
				\footnotesize la ligne de fuite};
			
			\clip (0, 0) rectangle (12, 6);
			\foreach \i in {-4, -2, ..., 10} \draw[green] (12, 6) -- (\i, 0);
			\foreach \i in {4, 8, ..., 24} \draw[blue] (-12, 6) -- (\i, 0);
			\end{tikzpicture}
		\end{figure}
		$$ {\color{blue} B} = \dfrac{{\color{green} d_C} {\color{red} A} + {\color{red} d_A} {\color{green} C}}{{\color{red} d_A} + {\color{green} d_C}} \qquad {\color{blue} d_B} = 2 \dfrac{{\color{red} d_A} {\color{green} d_C}}{{\color{red} d_A} + {\color{green} d_C}} $$
	\end{frame}

	\section{Détection des lignes}
	
	\begin{frame}{Les étapes}
		\begin{itemize}
			\small \item Changement d'espace de couleurs (RGB $\rightarrow$ CIE Lab)
			\small \item Lissage / élimination du bruit
			\small \item Calcul du gradient
			\small \item Transformée de Hough
			\small \item Regroupement des lignes et calcul des points de fuites
		\end{itemize}
		\centering
		\includegraphics[scale=0.33]{hooch1.jpg}
		\includegraphics[scale=0.4]{hooch2.jpg}
	\end{frame}

	\begin{frame}{CIE Lab}
		Besoin d'un espace dans lequel la distance entre deux couleur est en accord avec l'œil humain :
		$$ \Delta E = \sqrt{(R_1 - R_2)^2 + (V_1 - V_2)^2 + (B_1 - B_2)^2} \qquad \text{\huge \color{red} \text{\faThumbsODown}} $$
		$$ \Delta E = \sqrt{(L_1^* - L_2^*)^2 + (a_1^* - a_2^*)^2 + (b_1^* - b_2^*)^2} \qquad \text{\huge \color{green} \text{\faThumbsOUp}}  $$
	\end{frame}

	\begin{frame}{Filtre gaussien}
		\centering
		\begin{tikzpicture}[scale=0.8]
		\node at (0, 0.9) {\includegraphics[width=8cm]{smoothing.jpg}};
		\node[below] at (0, 0) {\small Filtre gaussien};
		\node[below] at (-3.45, 0) {\small Image original};
		\fill[white] (1.7, 0) rectangle (5.1, 2);
		\fill[white] (-5.1, 0) rectangle (-8.6, 2);
		\end{tikzpicture}
		$$ I'(z) = \dfrac{1}{W} \sum_{z' \in \Omega_{z}} I(z') \exp \left( - \dfrac{\| z - z'\|^2}{2 \sigma^2} \right) $$
		\small Où $\Omega_{z}$ est une fenêtre autour de $z \in \N^2$ et $W = \sum_{z' \in \Omega_{z}} \exp \left( - \frac{\| z - z'\|^2}{2 \sigma^2} \right)$.
	\end{frame}

	\begin{frame}{Filtre bilatéral}
		\centering
		\begin{tikzpicture}[scale=0.8]
		\node at (0, 0.9) {\includegraphics[width=8cm]{smoothing.jpg}};
		\node[below] at (0, 0) {\small Filtre gaussien};
		\node[below] at (-3.45, 0) {\small Image original};
		\node[below] at (3.45, 0) {\small Filtre Bilatéral};
		\end{tikzpicture}
		$$ I'(z) = \dfrac{1}{W} \sum_{z' \in \Omega_{z}} I(z) \, f(z, z') \, g \left( I(z), I(z') \right) $$
		\small Où $\Omega_{z}$ est une fenêtre autour de $z$ et $W = \sum_{z' \in \Omega_{z}} f(z, z') g(I(z), I(z'))$.
		$$ f(z, z') = \exp \left( - \dfrac{\| z - z'\|^2}{2 \sigma_f^2} \right) \qquad g(u, u') = \exp \left( - \dfrac{\| u - u'\|^2}{2 \sigma_g^2} \right) $$
	\end{frame}

	\begin{frame}{Sélection de zone intéressante}
		\begin{figure}[h]
			\centering
			\begin{tikzpicture}[scale=0.75]
			\node (a) at (0, 0) {\includegraphics[scale=0.75]{zone_select.png}};
			\node (b) at (8.5, 0) {\includegraphics[scale=0.9]{bilateral.png}};
			\draw[->, ultra thick] (a) -- node[above, align=center]
			{\footnotesize Coupe et \\[-1mm]
				\footnotesize application du \\[-1mm]
				\footnotesize filtre bilatéral} (b);
			\end{tikzpicture}
		\end{figure}
	\end{frame}

	\begin{frame}{Gradient (filtre de Sobel)}
		$$ G_{C, x} = \begin{bmatrix} -3 & 0 & 3 \\ -10 & 0 & 10 \\ -3 & 0 & 3 \end{bmatrix} \ast C
		\qquad G_{C, y} = \begin{bmatrix} -3 & -10 & -3 \\ 0 & 0 & 0 \\ 3 & 10 & 3 \end{bmatrix} \ast C
		$$
		$$ G_C = \sqrt{G_{C, x}^2 + G_{C, y}^2} $$
		\begin{block}{Amplitude du gradient}
			\vspace{-2mm}
			$$ G = \sqrt{G_{L^*} + G_{a^*} + G_{b^*}} $$
		\end{block}
	\end{frame}

	\begin{frame}{Gradient (filtre de Sobel)}
		On définie $(G_{C, x}', \, G_{C, y}')$ égal à $(G_{C, x}, \, G_{C, y})$ si $G_{C, x} \geqslant 0$ ou égal à $(-G_{C, x}, \, -G_{C, y})$ sinon. \vspace{3mm}
		
		\begin{block}{Angle du gradient}
			\vspace{-5mm}
			$$ \Theta = \arg \left( \left( \sum_{C \in \{L^*, a^*, b^*\}} G_C G_{C, x}' \right) + i \left( \sum_{C \in \{L^*, a^*, b^*\}} G_C G_{C, y}' \right) \right) $$
		\end{block}
	\end{frame}

	\begin{frame}{Gradient (filtre de Sobel)}
		\begin{figure}
			\centering
			\begin{tikzpicture}
			\node at (0, 0) {\includegraphics[scale=0.95]{bilateral2.png}};
			\node at (0, -3) {\includegraphics[scale=0.95]{sobel0.png}};
			\end{tikzpicture}
		\end{figure}
	\end{frame}

	\begin{frame}{Gradient (nettoyage)}
		\begin{figure}[h]
			\centering
			\begin{tikzpicture}[scale=0.8]
			\node (a) at (0, 0) {\includegraphics[scale=2.2]{sobel_wc.png}};
			\node (b) at (7.25, 0) {\includegraphics[scale=2.2]{sobel_c.png}};
			\draw[->, ultra thick] (a) -- node[above] {\small Nettoyage} (b);
			\node[align=center] (c) at (6, -2.2) {\footnotesize Effacé car la composante \\[-1mm]
				\footnotesize connexe est trop petite};
			\draw[->, thick, gray] (c.174) -- (0.98, -0.35);
			\node[align=center] (d) at (-0.3, -2.1) {\footnotesize Effacé car l'intensité \\[-1mm]
				\footnotesize est trop petite};
			\draw[->, thick, gray] (d.north) -- (-0.85, -0.5);
			\end{tikzpicture}
		\end{figure}
	\end{frame}

	\begin{frame}{Gradient (lissage)}
	\begin{algorithm}[H]
		\caption{Lissage de gradient}
		\begin{algorithmic}
			\State $M \gets \begin{bmatrix}
			0.0925 & 0.12 & 0.0925 \\
			0.12 & 0.15 & 0.12 \\
			0.0925 & 0.12 & 0.0925
			\end{bmatrix}$
			\Function{SmoothGrad}{$G, \Theta, x, y$}
			\State $a, b \gets 0, 0$
			\For{$i, j \in \{ -1, 0, 1 \}^2$}
			\State $coef \gets M[j+1][i+1] \times G[x+i][y+j]$
			\State $a \gets a + coef \times \cos \left( 2 \times \Theta[x+i][y+j] \right)$
			\State $b \gets b + coef \times \sin \left( 2 \times \Theta[x+i][y+j] \right)$
			\EndFor
			\State $\Theta[x][y] \gets \arg \left( a + ib \right) \, / \, 2$
			\EndFunction
		\end{algorithmic}
	\end{algorithm}
	\end{frame}

	\begin{frame}{Gradient (lissage)}
		\begin{figure}[h]
			\centering
			\begin{tikzpicture}
			\node[] (a) at (0, 0) {\includegraphics[scale=3.3]{sobel_ws.png}};
			\node[] (b) at (0, -3) {\includegraphics[scale=3.3]{sobel_s.png}};
			\node[below=3mm, white] at (a.150) {\small Sans lissage};
			\node[below=3mm, white] at (b.150) {\small Avec lissage};
			\end{tikzpicture}
		\end{figure}
	\end{frame}

	\begin{frame}{Gradient (nettoyage manuel)}
		\begin{figure}[h]
			\centering
			\begin{tikzpicture}
			\node (a) at (0, 0) {\includegraphics[scale=0.21]{w_man_c.png}};
			\node (b) at (6.6, 0) {\includegraphics[scale=0.21]{man_c.png}};
			\draw[->, ultra thick] (a) -- node[above] {\small coup de brosse} (b);
			\end{tikzpicture}
		\end{figure}
	\end{frame}

	\begin{frame}{Transformée de Hough (paramétrisation)}
		\begin{figure}[h]
			\centering
			\begin{tikzpicture}[thick]
			\draw[red] (2.093, -0.9) -- node[below] {\small $1$} (1.57, -0.9)
						-- node[left] {\small $|a|$} (1.57, -0.1);
			\draw[green] (0.5, 0) arc(0:33.7:0.5) node[right, pos=0.7] {\small $\theta$};
			\draw[green] (0, 0) -- node[above] {\small $\rho$} (1.05, 0.7);
			\draw[->] (0, -1) -- node[right, pos=0.95] {\small $y$} (0, 3);
			\draw[->] (-1.5, 0) -- node[above, pos=0.95] {\small $x$} (3, 0);
			\draw[very thick, blue] (-0.5, 3) -- (2.17, -1);
			\node[red] at (0, 2.25) {$\bullet$};
			\node[red, right] at (0, 2.25) {\small $b$};
			\node at (5.8, 2.5) {$y = {\color{red} a} x + {\color{red} b} \quad \text{\color{red} \faThumbsODown}$};
			\node at (5.8, 1.5) {${\color{green} \rho} = x \cos {\color{green} \theta} + y \sin {\color{green} \theta} \quad \text{\color{green} \faThumbsOUp}$};
			\node at (5.8, 0.5) {${\color{green} \rho} \in \left[0; \, D\right] \quad {\color{green} \theta} \in \left[-\pi / 2; \, \pi\right]$};
			\end{tikzpicture}
		\end{figure}
	\end{frame}

	\begin{frame}
		\begin{algorithm}[H]
			\caption{Transformée de Hough}
			\begin{algorithmic}
				\Function{Hough}{$G, \Theta$}
				\State $H^* \gets $ Null image of size $R \times T$
				\For{$(x, y) \in [0, W) \times [0; H) $ with $ G[x][y] > 0$}
				\For{$t \in [0; T)$}
				\State $\theta \gets 3 \pi / 2 \times t / T - \pi / 2$
				\State $\rho \gets x . \cos(\theta) + y . \sin(\theta)$
				\If{$\rho \geqslant 0$}
				\State $r \gets \rho \times R / D$
				\State $H^*[r][t] \gets H^*[r][t] + f(G, \Theta, \theta, x, y)$
				\EndIf
				\EndFor
				\EndFor
				\State \Return $H^*$
				\EndFunction
			\end{algorithmic}
		\end{algorithm}
	\end{frame}

	\begin{frame}{Transformée de Hough}
		$$ H^*[r][t] \gets H^*[r][t] + f(G, \Theta, \theta, x, y) $$
		\vspace{4mm}
		\begin{block}{Fonction $f$ utilisée}
			\vspace{-4mm}
			$$ f(G, \Theta, \theta, x, y) = \left( 1 - \left| \sin \left( \theta - \Theta[x][y] \right) \right|^{\frac{2}{3}}\right) \times \left( 1 + \frac{3}{2} \dfrac{G[x][y]}{\max (G)} \right) $$
		\end{block}
	\end{frame}

	\begin{frame}{Transformée de Hough}
		\begin{figure}[h]
			\centering
			\begin{tikzpicture}[scale=0.8]
			\draw[->, thick] (0, 0) -- (4, 0) node[below] {x};
			\draw[->, thick] (0, 0) -- (0, 4) node[left] {y};
			\draw[very thick, red] (-0.2, 1.87) -- (3, 4);
			\node[red] (a) at (0.66, 2.44) {\huge $+$};
			\node[red] (b) at (1.5, 3) {\huge $+$};
			\node[red] (c) at (2.33, 3.55) {\huge $+$};
			\node[above left, red] at (a) {$A$};
			\node[above left, red] at (b) {$B$};
			\node[above left, red] at (c) {$C$};
			\draw[very thick, green] (-0.2, 1) -- (4, 1);
			\node[green] (a) at (1, 1) {\huge $\times$};
			\node[green] (b) at (3, 1) {\huge $\times$};
			\node[above=1mm, green] at (a) {$D$};
			\node[above=1mm, green] at (b) {$E$};
			
			\draw[->, thick] (7, 0) -- (11.8, 0) node[below] {$\rho$};
			\draw[->, thick] (7, 0) -- (7, 4) node[left] {$\theta$};
			\draw[variable=\t, domain=-0.264:2.877, smooth, thick, red]
			plot({7 + 0.66 * cos(\t*180/3.142) + 2.44 * sin(\t*180/3.142)}, {(\t + 1.571) / 4.713 * 4});
			\draw[variable=\t, domain=-0.463:2.677, smooth, thick, red]
			plot({7 + 1.5 * cos(\t*180/3.142) + 3 * sin(\t*180/3.142)}, {(\t + 1.571) / 4.713 * 4});
			\draw[variable=\t, domain=-0.580:2.560, smooth, thick, red]
			plot({7 + 2.33 * cos(\t*180/3.142) + 3.55 * sin(\t*180/3.142)}, {(\t + 1.571) / 4.713 * 4});
			\node[red] at (9.75, 2.4) {A};
			\node[red] at (10.6, 2.2) {B};
			\node[red] at (11.5, 2) {C};
			\draw[variable=\t, domain=-0.785:2.356, smooth, thick, green]
			plot({7 + 1 * cos(\t*180/3.142) + 1 * sin(\t*180/3.142)}, {(\t + 1.571) / 4.713 * 4});
			\draw[variable=\t, domain=-1.249:1.892, smooth, thick, green]
			plot({7 + 3 * cos(\t*180/3.142) + 1 * sin(\t*180/3.142)}, {(\t + 1.571) / 4.713 * 4});
			\node[green] at (8.1, 2) {D};
			\node[green] at (9, 0.55) {E};
			
			\draw[->, thick] (6.2, 1.5) node[left] {ligne} -- (7.92, 2.62);
			\draw[->, thick] (10.2, 3.7) node[right] {ligne} -- (8.83, 3.23);
			\node at (2, -0.42) {\small Espace de l'image};
			\node at (9.4, -0.42) {\small Espace de Hough};
			\end{tikzpicture}
		\end{figure}
	\end{frame}

	\begin{frame}
		\begin{figure}[h]
			\makebox[\textwidth][c]{\begin{tikzpicture}[scale=0.55]
			\node (a) at (0, 0) {\includegraphics[scale=0.6325]{sobel_final.png}};
			\node (b) at (8.3, 0) {\includegraphics[scale=0.6325]{hough_painting.png}};
			\node[left=-1mm] at (a.west) {\footnotesize Gradient};
			\node[right=-1mm, align=center] at (b.east) {\footnotesize Transformée \\
													\footnotesize de Hough};
			\draw[->, ultra thick] (a) -- (b);
			\node (c) at (4.15, -6.25) {\includegraphics[scale=0.726]{lines0.png}};
			\node[left=-1mm, align=center] at (c.west) {\footnotesize Reconstruction \\
				\footnotesize des lignes};
			\draw[->, thin, red] (10.8, -0.42) -- (7.8, -9.34);
			\draw[->, thin, red] (10.555, -0.416) -- (7.33, -9.12);
			\draw[->, thin, red] (10.04, -0.417) -- (7.15, -8.5);
			\draw[->, thin, red] (5.95, 1.34) -- (2.8, -3.2);
			\draw[->, thin, red] (6.06, -1.598) -- (1.35, -4.6);
			\end{tikzpicture}}
		\end{figure}
	\end{frame}

	\begin{frame}{Résultat}
		\begin{figure}[h]
			\centering
			\includegraphics[width=\linewidth]{final.png}
		\end{figure}
	\end{frame}

	\section{Redessiner avec une perspective parfaite}
	
	\begin{frame}{Objectif}
		\begin{itemize}
			\small \item Calculer de nouvelles lignes de carrelage qui se croisent toutes exactement en un seul point de fuite et restent aussi près que possible des lignes précédemment obtenues.
			\small \item Modifier le gradient de l'image pour que les pixels de haute intensité soient sur les nouvelles lignes.
			\small \item Utiliser du Poisson editing [OBBT07] [PGB03] pour obtenir une nouvelle image avec une perspective parfaite à partir du nouveau gradient.
		\end{itemize}
		{\tiny [OBBT07] Alexandrina Orzan, Adrien Bousseau, Pascal Barla, and Joëlle Thollot. Structure-
		preserving manipulation of photographs. In \textit{Proceedings of the 5th International Symposium on Non-Photorealistic Animation and Rendering 2007, San Diego, California, USA, August 4-5, 2007}, pages 103–110, 2007. }
		{\tiny [PGB03] Patrick Pérez, Michel Gangnet, and Andrew Blake. Poisson image editing. \textit{ACM Trans. Graph.}, 22(3):313–318, 2003. }
	\end{frame}

	\begin{frame}{Variables à optimiser}
		Perspective paramétrée par les variables :
		$$ {\color{red} x_A, \, y_A, \, d_A}, {\color{blue} x_B, \, y_B, \, d_B}, {\color{green} x_C, \, y_C, \, d_C} $$
		Équations permettant d'éliminer $\color{blue} x_B, \, y_B, \, d_B$ :$$ {\color{blue} B} = \dfrac{{\color{green} d_C} {\color{red} A} + {\color{red} d_A} {\color{green} C}}{{\color{red} d_A} + {\color{green} d_C}} \qquad {\color{blue} d_B} = 2 \dfrac{{\color{red} d_A} {\color{green} d_C}}{{\color{red} d_A} + {\color{green} d_C}} $$
		Variables restantes :
		$$ {\color{red} x_A, \, y_A, \, d_A}, {\color{green} x_C, \, y_C, \, d_C} $$
	\end{frame}

	\begin{frame}{Fonctions à minimiser}
		Afin d'avoir une ligne de fuite horizontale et des carreaux bien orientés on minimise :
		$$ R = \dfrac{({\color{green} y_C} - {\color{red} y_A})^2}{({\color{green} x_C} - {\color{red} x_A})^2 + ({\color{green} y_C} - {\color{red} y_A})2} \times \left( v . ({\color{green} d_C} - {\color{red} d_A})^2 + c^{te} \right) $$
	\end{frame}

	\begin{frame}{Fonctions à minimiser}
		Afin que les points de fuite restent à leur place :
		$$ \begin{matrix}
		S & = & v . ({\color{red} x_A} - x_{A_0})^2 + v . ({\color{red} y_A} - y_{A_0})^2 \\
		& + & v . ({\color{blue} x_B} - x_{B_0})^2 + v . ({\color{blue} y_B} - y_{B_0})^2 \\
		& + & v . ({\color{green} x_C} - x_{C_0})^2 + v . ({\color{green} y_C} - y_{C_0})^2
		\end{matrix} $$
	\end{frame}

	\begin{frame}{Fonctions à minimiser}
		\begin{figure}[h]
			\makebox[\textwidth][c]{\begin{tikzpicture}[scale=0.64]
			\node at (1.5, 4.9) {\small Lines from the previous part};
			\draw (-2, 4.5) -- (-2, 0) -- (5, 0) -- (5, 4.5);
			\draw[blue, thick] (-3, 3.5) -- (6, 4.5) ;
			\node[red] at (0, 3.83) {$\bullet$};
			\node[red, above] at (0, 3.83) {$A_0$};
			\node at (0.657, 0) {$\bullet$};
			\node[below] at (0.657, 0) {$P_{A_0}$};
			\node at (5, 0.651) {$\bullet$};
			\node[right] at (5, 0.651) {$P'_{A_0}$};
			\node[red, below] at (3, 0) {\footnotesize $n_A = 8 \text{ lines}$};
			
			\begin{scope}
			\clip (-2, 4.5) rectangle (5, 0);
			\foreach \i in {0, ..., 7} \draw[red] (0, 3.83) -- ({0.994*\i + 1}, {0.11*\i - 2});
			\end{scope}
			
			\node[blue] at (6.35, 4.1) {\tiny Vanishing Line};
			\node[] at (6.35, 2.25) {\tiny Image Border};
			
			\node at (10.95, 4.9) {\small New lines};
			\draw (7.7, 4.5) -- (7.7, 0) -- (14.7, 0) -- (14.7, 4.5);
			\draw[blue, thick] (6.7, 3.85) -- (15.6, 4.45);
			\draw[blue, thick] (6.7, 0.4) --
			node[pos=0.05, above] {\footnotesize $(D)$}
			node[pos=0.55, sloped, above] {\tiny
				$\color{black} ({\color{red} n_A} - 1) \times {\color{purple} d_A}$}
			(15.7, 1);
			\draw (8.7, 3.983) -- node[left, below, sloped] {\scriptsize $d_{trans}$} (8.929, 0.549);
			\node[purple] at (8.95, 4) {$\bullet$};
			\node[purple, above] at (8.95, 4) {$A$};
			\node at (10.357, 0) {$\bullet$};
			\node[below] at (10.357, 0) {$P_{A_0}$};
			\node at (14.7, 0.651) {$\bullet$};
			\node[right] at (14.7, 0.651) {$P'_{A_0}$};
			
			\node[purple] at (10.136, 0.629) {$\bullet$};
			\node[purple, above left] at (10.136, 0.629) {$P_A$};
			\node[purple] at (13.808, 0.874) {$\bullet$};
			\node[purple, above] at (13.808, 0.874) {$P'_A$};
			
			\draw[ultra thick, brown, ->] (13.45, 4.3) -- node[above] {\footnotesize $\vec{u}$} (14.2, 4.35);
			\draw[ultra thick, brown, ->] (13.45, 4.3) -- node[left] {\footnotesize $\vec{n}$} (13.5, 3.55);
			
			\draw[purple] (8.95, 4) -- (10.357, 0);
			\draw[purple] (8.95, 4) -- (14.7, 0.3);
			\end{tikzpicture}}
			\vspace{-10mm}
		\end{figure}
	
		$$ T = dist \left( ({\color{red}A} {\color{red}P'_A}), \, P'_{A_0} \right)^2 + dist \left( ({\color{blue}B} {\color{blue}P'_B}), \, P'_{B_0} \right)^2 + dist \left( ({\color{green}C} {\color{green}P'_C}), \, P'_{C_0} \right)^2 $$
	\end{frame}

	\begin{frame}{Minimisation}
		On utilise la méthode Newton pour minimiser :
		$$ F(x_A, y_A, x_C, y_C, d_A, d_C) = c^{te} . R \, + \, v . S \, + \, v . T $$
	\end{frame}

	\begin{frame}{Résultat}
		\begin{figure}[h]
			\centering
			\includegraphics[width=\linewidth]{perf.jpg}
		\end{figure}
	\end{frame}

\end{document}