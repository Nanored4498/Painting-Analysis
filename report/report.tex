\documentclass{article}

\usepackage[english]{babel}
\usepackage[utf8]{inputenc}
\usepackage{kpfonts}
\usepackage[left=2.8cm, right=2.8cm, top=2.8cm, bottom=3cm]{geometry}
\usepackage{titling}
\usepackage{graphicx}
\usepackage{hyperref}

\graphicspath{{ims/}}

%%%%%%%%%%%% Title Part %%%%%%%%%%%%
\pretitle{
	\begin{center}
	\includegraphics[width=4cm]{logo_ens_lyon.pdf}
	\hspace{5cm}
	\includegraphics[width=5.5cm]{logo_tu_delft.png}
	\LARGE
}
\title{
	\rule{\linewidth}{0.4mm}
	\textbf{Perspective Check in Paintings} \\
	\textit{M1 Internship}
	\rule{\linewidth}{0.6mm}
}
\author{
	Yoann Coudert--Osmont \\[3mm]
	\textit{Supervised by} \\
	Elmar Eisemann \qquad Ricardo Marroqium \\[2mm]
}
\date{May - July 2019}
%%%%%%%%%%%%%%%%%%%%%%%%%%%%%%%%%%%%

\begin{document}
	\maketitle
	
	\section{Introduction}
	
	In a Delft museum, paintings by Pieter de Hooch are on display. It is quite easy to notice that pairs of paintings are very similar. One hypothesis that would explain this strangeness is the possibility that another painter copied Pieter de Hooch's paintings. Looking at the pairs in more detail, we can notice that quite often one painting respects the rules of perspective well and the other does not respect them. This gives credibility to the hypothesis of the existence of another painter. But the human tends to be biased to check that in each pair one painting respects the rules of perspective and the other does not. Then comes the need to create a tool to check whether the perspective is respected in a table with as little human intervention as possible.
	
	\section{Conclusion}
	
	As a conclusion I did nothing during this internship !
	
	\appendix
	
	\section{Source Code}
	
	Here is my git repository: \url{https://github.com/Nanored4498/Painting-Analysis}.
	
\end{document}